\section{Related Work}

The Related Work section of your report should provide a review of recent literature in the area of your research. This is distinguished from the Background section because it is typically newer and more experimental. If there are standard terms or techniques mentioned in the literature, then you can define what these are in the Background and use the Related Work section to explain how researchers have used the standard techniques as benchmarks or fundamental methodologies for their research. For example, if you review an article that describes using k-means clustering for finding appropriate groups of patients with similar sets of symptoms, then you could describe what k-means clustering is in your Background section and describe how researchers used that technique on patient data in your Related Work section. When you review literature, be sure to explain how the articles you cite are relevant to your project. Be critical---outline pros and cons of the work you are reviewing. Be clear to explain how the work you review is different from your own work. Note that you may find it easier to compare and contrast others' techniques with yours later in the report, after you have explained your own work. That is fine---just be sure to forward reference in the Related Work where you will compare to your own work (and backward reference in the later sections back to the Related Work). This can include information that you had in your Project Proposal report that was due in April, but should typically be substantially expanded from what you had in your proposal.

Refer to links and resources on the KEATS page to help with your literature review.
Be sure to provide complete references when someone's work is mentioned.

Examples of articles we might cite are~\cite{Doe11} and~\cite{JohSil05}.
